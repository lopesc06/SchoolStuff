%-------------------------------------- COMIENZA descripción del caso de uso.
\begin{UseCase}{CU7}{Consultar Rutas}{
	Cuando se requiere conecer alguna de las diferentes rutas se debera acceder al sistema y validar los priviligios para poder ver esta informacion.
	Se podra consultar alguna ruta apartir de su clave identificadora, ya que con esta se mostrará las direcciones de las rutas y los puntos a los cuales se debe ir.
}
	%---------------------------------------------------------
	% Datos de gestión del CU.
	\UCitem{Versión}{0.1}
	\UCitem{Autor}{Ivan Baltazar Camacho}
	\UCitem{Modificado}{09 - Nov - 2015}
	\UCitem{Revisor}{Juan Daniel López Santiago}
	\UCitem{Revisado}{09 - Nov - 2015}
	\UCitem{Status}{Revisado}
	\UCitem{Aprobado}{Pendiente por aprobar}
	\UCitem{Madurez}{Baja}
	\UCitem{Volatilidad}{Baja}
	\UCitem{Dificultad}{Alta}
	\UCitem{Prioridad}{Alta}
	
	%---------------------------------------------------------
	% Datos del CU
	\UCitem{Actor}{Jefe de Unidad Departamental de la DPASR, Director de la DPASR,publico general 1er Asesor Tecnico de la DPASR y Subdirector de Sistemas
}
	\UCitem{Propósito}{Mantener control sobre las rutas de recolección de sólidos.}
	\UCitem{Entradas}{numero de ruta,delegacion o Estado}
	\UCitem{Salidas}{Delegación, Hora de recolección, Estado, Colonia ,Punto específico y Plan de manejo,Pantalla de notificación de ruta no encontrada o inexistente.}
	\UCitem{Origenes}{Teclado y mouse.}
	\UCitem{Destinos}{Pantalla y/o impresora}
	\UCitem{Precondiciones}{Debe de existir una ruta}
	\UCitem{Postcondiciones}{se podra visualizar los datos de una ruta y estos se podran imprimir en un pequeño informe}
	\UCitem{Errores}{Que no exista la ruta que se busca.}
	\UCitem{Fuentes}{Manual de Usuario SIRS e Inventario de Residuos Sólidos (IRS)}
	\UCitem{Observaciones}{¿Por qué estado?, el plan de manejo corresponde a un sitio específico no a una ruta.}
\end{UseCase}

\begin{UCtrayectoria}{Principal}
	\UCpaso El usuario presiona el botón consultar rutas.
	\UCpaso	El sistema abrirá la interfaz de consultar rutas.
	\UCpaso	El sistema  mostrara un conjunto de cajas de texto para seleccionar opciones de busqueda.
	\UCpaso	El usuario podrá elegir como buscar.
	\UCpaso	El sistema buscara las rutas relacionadas con la peticion del usuario .
	\UCpaso	el sistema mostrara una lista de las rutas encontradas.
	\UCpaso	El usuario podrá seleccionar una ruta o podra repetir los pasos de busqueda desde el punto 4. 
	\UCpaso	El sistema mostrara la informacion realcionada a la ruta que se selecciono .
	\UCpaso El usuario podra imprimer la informacion precionando en el boton imprimir.
	\UCpaso	El usuario podra terminar precionando en el boton finalizar.
	\UCpaso	FIN
\end{UCtrayectoria}
	
\begin{UCtrayectoriaA}{A}{Ruta no encontrada.}
	\UCpaso El sistema validara la operación.
	\UCpaso	El sistema no encuentra ruta.
	\UCpaso	El sistema muestra mensaje “no se encontraron rutas”.
	\UCpaso	El usuario presionara aceptar para cerrar mensaje.
	\UCpaso	El capturista podrá modificar la información desde el punto 4 de la trayectoria principal.
	\UCpaso	Fin de trayectoria.
\end{UCtrayectoriaA}

\begin{UCtrayectoriaA}{B}{No selecciono datos de busqueda.}
	\UCpaso El sistema validara la operación.
	\UCpaso	El sistema no encuentra datos para buscar.
	\UCpaso	El sistema muestra mensaje “Falta información requerida”.
	\UCpaso	El usuario presionara aceptar para cerrar el mensaje.
	\UCpaso	El usuario podrá proporcionar informacion desde el  punto 4 de la trayectoria principal.
	\UCpaso	Fin de trayectoria.
\end{UCtrayectoriaA}



		
%-------------------------------------- TERMINA descripción del caso de uso.