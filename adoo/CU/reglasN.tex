%---------------------------------------------------------
\section{Hechos del Negocio}

\begin{description}
	\item[Créditos {\em de} un Estudiante:] Es la suma de los Créditos de las Materias aprobadas por el Estudiante.
	\item[Materia {\em aprobada}:] Materia que el Estudiante haya Inscrito y aprobado.
	\item[Materias {\em inscritas} por un Estudiante] Un estudiante se registra a un grupo de una Materia a fin de cursarla y aprobarla.
	\item[{\em Cupo} de un grupo:] Todos los grupos no pueden tener mas de 30 alumnos por la capacidad de las aulas.
\end{description}

%---------------------------------------------------------
\section{Reglas de Negocio}

\begin{BussinesRule}{BR129}{Determinar si un Estudiante puede inscribir Seminario.} 
	\BRitem[Descripción:] Un Estudiante requere del 80\% de créditos para inscribirse a un Seminario y no haber cursado y reprobado otro seminario.
	\BRitem[Tipo:] Restricción de integridad.
	\BRitem[Nivel:] Obligatorio.
\end{BussinesRule}

\begin{BussinesRule}{BR130}{Determinar si un Estudiante puede inscribirse en un Seminario}
	\BRitem[Descripción:] El Estudiante debe pertenecer a la Carrera del Seminario y debe haber Cupo en el grupo del Seminario.
	\BRitem[Tipo:] Restricción de operación y de integridad.
	\BRitem[Nivel:] Obligatorio.
\end{BussinesRule}

\begin{BussinesRule}{BR143}{Validar el horario del estudiante}
	\BRitem[Descripción:] Las Materias y Seminarios inscritos por el alumno, en un periodo específico, no pueden impartirse en el mismo día de la semana en horas traslapadas.
	\BRitem[Tipo:] Restricción de operación.
	\BRitem[Nivel:] Obligatorio.
\end{BussinesRule}

\begin{BussinesRule}{BR180}{Calcular costos del Estudiante}
	\BRitem[Descripción:] Los servicios se cobran de la siguiente forma:
		\begin{Citemize}
			\item {\em Estudiantes Regulares:} Se les Cobran todos los servicios al 100\% de su costo.
			\item {\em Estudiantes becados:} Se les otorga un 80\% de descuento en el costo de todos los servicios (antes del IVA).
			\item {\em Estudiantes extranjeros:} Se les cobran los servicios al 200\% del costo registrado.
		\end{Citemize}
	\BRitem[Sentencia:] $\forall~e~\in~\mathbb{E}\textrm{studiantes}~\land~\forall~s~\in \mathbb{S}\textrm{eminario}~\Rightarrow$
		\begin{displaymath}
			Costo(e,s) = \left\{ \begin{array}{ll}
			s.costo & , si~e.tipo = \textrm{Estudiante regular}\\
			{s.costo}\over{5} & , si~e.tipo = \textrm{Estudiante becado}\\
			s.costo \cdot 2 & , si~e.tipo = \textrm{Estudiante extranjero}
			\end{array} \right.
		\end{displaymath}

	\BRitem[Tipo:] Cálculo.
	\BRitem[Nivel:] Obligatorio.
\end{BussinesRule}

\begin{BussinesRule}{BR45}{Calcular impuestos por seminario}
	\BRitem[Descripción:] Los impuestos corresponden al 16\% correspondientes al IVA.
	\BRitem[Sentencia:] $Impuesto(e, s) = Costo(e, s)\cdot0.16$.
	\BRitem[Tipo:] Cálculo.
	\BRitem[Nivel:] Obligatorio.
\end{BussinesRule}

\begin{BussinesRule}{BR100}{Recibo del Estudiante por inscripción a Seminario.}
	\BRitem[Descripción:] El  Recibo del Estudiante debe mostrar el total del costo con el siguiente desglose:
		\begin{displaymath}\begin{array}{lr}
			Costo: & \$ XXX.XX\\
			Descuento~aplicado~(YY\%): & \$ XXX.XX\\
			Subtotal: & \$ XXX.XX\\
			IVA~(16\%): & \$ XXX.XX\\\hline
			Total: & \$ XXX.XX
		\end{array}\end{displaymath}
	\BRitem[Sentencia:] $CostoTotal = Costo(e, s) + Impuesto(e, s)$.
	\BRitem[Tipo:] Restricción de operación/Cálculo.
	\BRitem[Nivel:] Obligatorio.
\end{BussinesRule}



