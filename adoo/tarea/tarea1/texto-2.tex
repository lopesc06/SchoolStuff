\documentclass{article} 

\usepackage[utf8]{inputenc}
\usepackage{graphicx}
\usepackage[hidelinks]{hyperref} 

\renewcommand{\baselinestretch}{1.5}

\title{\Huge ADOO.\vspace{1 cm}  \\ ¿Qué es software? \vspace{1 cm} }


\author{\\ \Huge Arturo Escutia López.\vspace{2 cm}\\ \Huge Profesor:Ulises Velez Saldaña}
\date{\Huge\vspace{2 cm}Septiembre 14, 2015 \Huge  \vspace{2 cm} \\2CM12 \vspace{4 cm}}

\begin{document} 


\maketitle

 \LARGE  \textbf{Definición 1}\\
\large 
Según la RAE, el software es un conjunto de programas, instrucciones y reglas informáticas que permiten ejecutar distintas tareas en una computadora.
Se considera que el software es el equipamiento lógico e intangible de un ordenador. En otras palabras, el concepto de software abarca a todas las aplicaciones informáticas, como los procesadores de textos, las planillas de cálculo y los editores de imágenes.
El software es desarrollado mediante distintos lenguajes de programa.\\ 

 \LARGE  \textbf{ Definición 2 } \\
\large 
 Recurriendo al diccionario de informática publicado originalmente por la Oxford University Press (1993) el término software o programa se aplica a aquellos componentes de un sistema informático que no son tangibles, es decir, que físicamente no se pueden tocar. 
 el programa es sencillamente el conjunto de instrucciones que contiene la computadora, ya sean instrucciones para poner en funcionamiento el propio sistema informático (software de sistema) o instrucciones concretas dirigidas a programas particulares del usuario (software
específico). \\



 \LARGE  \textbf{ Definición 3 } \\
\large 
Es el conjunto de los programas de cómputo, procedimientos, reglas, documentación y datos asociados, que forman parte de las operaciones de un sistema de computación.
\\Extraído del estándar 729 del IEEE5\\


 \LARGE  \textbf{ Definición 4 }\\
\large 
Software es una secuencia de instrucciones que son interpretadas y/o ejecutadas para la gestión, redireccionamiento o modificación de un dato/información o suceso. 
Software también es un producto, el cual es desarrollado por la ingeniería de software, e incluye no sólo el programa para la computadora, sino que también manuales y documentación técnica.\\ 

\LARGE  \textbf{ Definición 5 }\\
\large 

El software se refiere a los programas y datos almacenados en un equipo. En otras palabras, son las instrucciones responsables de que el hardware (la máquina) realice su tarea.
El lenguaje utilizado por el software para comunicarse con el hardware es de tipo binario, viene en forma de instrucciones las cuales son ejecutadas por cada una de las partes del hardware (monitor, mouse, teclado, impresora, CPU, CD-ROM, disco duro, etc.)

\end{document}