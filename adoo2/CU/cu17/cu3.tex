%-------------------------------------- COMIENZA descripción del caso de uso.
\begin{UseCase}{CU3}{Dar de Alta Chofer}{
	Cuando se requeire registrar un chofer destinado a un vehiculo recolector de residuos, el actor podra entrar al sistema para que este llene un formulario con los datos de , nombre, apellido(s) CURP,No.licencia, telefono y asignarle un no.de conductor, el sistema mostrara los datos correspondietes y lo registrara en el sistema.


}
	%---------------------------------------------------------
	% Datos de gestión del CU.
	\UCitem{Versión}{0.3}
	\UCitem{Autor}{Miguel Ángel Medina Zarazua}
	\UCitem{Modificado}{27 - Nov - 2015}
	\UCitem{Revisor}{Ivan Baltazar Camacho}
	\UCitem{Revisado}{30 - Nov - 2015}
	\UCitem{Status}{Aprobado}
	\UCitem{Aprobado}{30 - Nov - 2015}
	\UCitem{Madurez}{Media}
	\UCitem{Volatilidad}{Baja}
	\UCitem{Dificultad}{Alta}
	\UCitem{Prioridad}{Alta}
	
	%---------------------------------------------------------
	% Datos del CU
	\UCitem{Actor}{1er Capturista de la DPASR}
	\UCitem{Propósito}{Llevar un control sobre los conductores que tiene asignado un(Varios) vehiculo(s) de recolección de residuos.}
	\UCitem{Entradas}{Nombre(s), Apellido Paterno o Materno,
		CURP, No. Licencia, telefono celular, teléfono casa, correo electrónico.}
	\UCitem{Salidas}{Se muestran los datos con lo que el conductor fue registrado as
		como un mensaje e registro exitoso.}
	\UCitem{Origenes}{Teclado y mouse.}
	\UCitem{Destinos}{Pantalla Y base de datos.}
	\UCitem{Precondiciones}{El conductor no debe estar registrado en el sistema.}
	\UCitem{Postcondiciones}{El conductor es registrado en el sistemas.}
	\UCitem{Errores}{Que el conductor ya exista en el sistema.}
	\UCitem{Fuentes}{Manual de Usuario SIRS.}
	\UCitem{Observaciones}{}
\end{UseCase}

\begin{UCtrayectoria}{Principal}
	\UCpaso[\UCactor] Introduce su Usuario  y Contraseña en el sistema vía teclado en la  \IUref{UI}{Pantalla  login de usuarios}.(Incluye CU Login Usuario).
	
	\UCpaso[\UCactor] Solicita dar de alta un Conductor seleccionando la opcion Nuevo chofer  del Menú Vehiculos recolectores.
	
	
	\UCpaso Solicita la información mostrando la \IUref{UI1}{Nuevo Conductor}.
	\UCpaso[\UCactor] Proporciona los campos para  los datos siguientes: CURP, No. Licencia, Nombre, Apellido Paterno, Apellido Materno, teléfono y correo electronico.
	\UCpaso Verifica que no existan campos requeridos vacíos  en aquellos campos que son obligatorios. \Trayref{A}.
	\UCpaso El sistema verifica que no exista registro de dicho conductor en el sistema. \Trayref{B}.
	\UCpaso El sistema verifica que el CURP tenga formato correcto. \Trayref{C}.
	\UCpaso El sistema agrega el registro del nuevo Conductor.
	\UCpaso El sistema notifica que la operación se realizó con éxito.
	\UCpaso Fin.
	
	
\end{UCtrayectoria}
	
\begin{UCtrayectoriaA}{A}{Campos requeridos vacíos.}
	\UCpaso El sistema identifica que existen campos requeridos vacíos.
	\UCpaso El sistema muestra el mensaje de "El Dato es obligatorio".
	\UCpaso Fin de trayectoria.
\end{UCtrayectoriaA}

\begin{UCtrayectoriaA}{B}{Conductor registrado anteriormente.}
	\UCpaso El sistema identifica que el Conductor ya está registrado en el sistema.
	\UCpaso El sistema muestra el mensaje "El conductor ya está registrado".
	\UCpaso El CU continua en el paso 3 de la trayectoria principal.
	\UCpaso Fin de trayectoria.
\end{UCtrayectoriaA}

\begin{UCtrayectoriaA}{C}{Formato de CURP incorrecto.}
	\UCpaso El sistema identifica que el CURP tiene formato incorrecto.
	\UCpaso El sistema muestra el mensaje "El formato del CURP es incorrecto".
	\UCpaso El CU continua en el paso 3 de la trayectoria principal.
	\UCpaso Fin de trayectoria.
\end{UCtrayectoriaA}



		
%-------------------------------------- TERMINA descripción del caso de uso.