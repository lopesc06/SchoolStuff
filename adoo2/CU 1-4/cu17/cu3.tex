%-------------------------------------- COMIENZA descripción del caso de uso.
\begin{UseCase}{CU3}{Dar de Alta Chofer}{
	Cuando se requeire registrar un chofer destinado a un vehiculo recolector de residuos, el actor podra entrar al sistema para que este llene un formulario con los datos de , nombre, apellido(s) CURP,No.licencia, telefono y asignarle un no.de conductor, el sistema mostrara los datos correspondietes y lo registrara en el sistema.


}
	%---------------------------------------------------------
	% Datos de gestión del CU.
	\UCitem{Versión}{0.1}
	\UCitem{Autor}{Miguel Ángel Medina Zarazua}
	\UCitem{Modificado}{20 - Mayo - 2016}
	\UCitem{Autor}{Arturo Escutia López}
\UCitem{Modificado}{20 - Mayo - 2016}
	\UCitem{Status}{Revisado}
	\UCitem{Aprobado}{Pendiente por aprobar}
	\UCitem{Madurez}{Media}
	\UCitem{Volatilidad}{Baja}
	\UCitem{Dificultad}{Alta}
	\UCitem{Prioridad}{Alta}
	
	%---------------------------------------------------------
	% Datos del CU
		\UCitem{Actor}{Director de Proyectos de Agua, Suelo y Residuos, Asesor técnico, Capturista ,Subdirector de Sistemas}
	\UCitem{Propósito}{Llevar un control sobre los conductores que tiene asignado un(Varios) vehiculo(s) de recolección de residuos.}
	\UCitem{Entradas}{No. de Conductor, Nombre(s), Apellido Paterno o Materno,
		CURP, o No. Licencia, telefono.}
	\UCitem{Salidas}{Se muestran los datos con lo que el conductor fue registrado as
		como un mensaje e registro exitoso.}
	\UCitem{Origenes}{Teclado y mouse.}
	\UCitem{Destinos}{Pantalla Y base de datos.}
	\UCitem{Precondiciones}{El conductor no debe estar registrado en el sistema.}
	\UCitem{Postcondiciones}{El conductor es registrado en el sistema.}
	\UCitem{Errores}{Que el conductor ya exista en el sistema.}
\UCitem{Fuentes}{Pagina de SMA}
	\UCitem{Observaciones}{No solo se debería validar longitud sino tambien que no tenga numeros y caracteres especiales, Mostrar  que campos estan vacios, No se indica a donde sale , ya que puede ser al mismo formulario o a la página principal, Validar ambos apellidos y no sólo el materno}
\end{UseCase}

\begin{UCtrayectoria}{Principal}
	\UCpaso[\UCactor] Solicita dar de alta un Conductor presionando el botón Registrar
	Conductor de la \IUref{UI5}{Gestionar Conductores}.
	\UCpaso Solicita la información mostrando la \IUref{UI1}{Registrar conductor}.
	\UCpaso[\UCactor] Proporciona los datos los datos siguientes: CURP, No. Licencia, Nombre, Apellido Paterno, Apellido Materno, teléfono, se
	le asigna un No. de conductor .
	\UCpaso Verifica que no existan campos requeridos vacios \Trayref{A}.
	\UCpaso El sistema verifica que no exista registro de dicho conductor en el sistema. \Trayref{B}.
	\UCpaso El sistema verifica que el CURP tenga formato correcto. \Trayref{C}.
	\UCpaso El sistema verifica que el No. Licencia tenga formato correcto. \Trayref{D}.
	\UCpaso El sistema verifica que Nombre, Apellido Paterno, Apellido Materno tengan longitud correcta. \Trayref{E}
	\UCpaso El sistema verifica que el teléfono tenga formato correcto. \Trayref{F}.
	\UCpaso El sistema agrega el registro del nuevo Conductor.
	\UCpaso El sistema notifica que la operación se realizó con éxito.
	\UCpaso Fin.
	
	
\end{UCtrayectoria}
	
\begin{UCtrayectoriaA}{A}{Campos requeridos vacíos.}
	\UCpaso El sistema identifica que existen campos requeridos vacíos.
	\UCpaso El sistema muestra el mensaje de "Campos vacíos requeridos".
	\UCpaso El capturista presiona el botón salir
	\UCpaso Fin de trayectoria.
\end{UCtrayectoriaA}

\begin{UCtrayectoriaA}{B}{Conductor registrado anteriormente.}
	\UCpaso El sistema identifica que el Conductor ya está registrado en el sistema.
	\UCpaso El sistema muestra el mensaje "El conductor ya está registrado".
	\UCpaso El CU continua en el paso 3 de la trayectoria principal.
	\UCpaso Fin de trayectoria.
\end{UCtrayectoriaA}

\begin{UCtrayectoriaA}{C}{Formato de CURP incorrecto.}
	\UCpaso El sistema identifica que el CURP tiene formato incorrecto.
	\UCpaso El sistema muestra el mensaje "CURP incorrecto".
	\UCpaso El CU continua en el paso 3 de la trayectoria principal.
	\UCpaso Fin de trayectoria.
\end{UCtrayectoriaA}

\begin{UCtrayectoriaA}{D}{Formato de No de Licencia incorrecto.}
	\UCpaso El sistema identifica que el No. de Licencia tiene formato incorrecto.
	\UCpaso El sistema muestra el mensaje "No. de Licencia incorrecto".
	\UCpaso El capturista presiona el botón “aceptar”
	\UCpaso. Fin de trayectoria.
\end{UCtrayectoriaA}

\begin{UCtrayectoriaA}{E}{Formato de Nombre incorrecto.}
	\UCpaso El sistema identifica que el nombre tiene longitud incorrecta.
	\UCpaso El sistema muestra el mensaje "Longitud de Nombre incorrecta".
	\UCpaso El CU continua en el paso 3 de la trayectoria principal.
	\UCpaso Fin de trayectoria.
\end{UCtrayectoriaA}

\begin{UCtrayectoriaA}{F}{Formato de Apellido(s) incorrecto(s)}
	\UCpaso El sistema identifica que el Apellido Materno y Apellido Paterno tiene longitud incorrecta y/o caracteres válidos.
	\UCpaso El sistema muestra el mensaje "Formato de Apellido(s) incorrecto".
	\UCpaso El CU continua en el paso 3 de la trayectoria principal.
	\UCpaso Fin de trayectoria.
\end{UCtrayectoriaA}

		
%-------------------------------------- TERMINA descripción del caso de uso.