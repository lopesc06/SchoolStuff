%--------------------------------------------------------------------
\section{Propósito}

El propósito de este documento es explicar el proceso llevado a cabo en el análisis y definición para la toma de requerimientos con respecto a la generación del inventario de residuos  sólidos
%--------------------------------------------------------------------
\section{Alcance}
El producto es un sistema web que permitirá al usuario acceder desde cualquier plataforma que cuente con un Navegador para gestionar información acerca del manejo de residuos sólidos.


%--------------------------------------------------------------------
\section{Definiciones y acrónimos}
\textbullet CMAP:Clasificación Mexicana de Actividades  y Productos del INEGI\\

\textbullet SMA:Secretaría del Medio Ambiente\\

\textbullet DGRVA:Dirección General de Regulación y Vigilancia Ambiental, perteneciente a la SMA\\

\textbullet DPASR:Dirección de Proyectos de Agua, Suelo y Residuos, perteneciente a la DGRVA\\

\textbullet DRA:Dirección de Regulación Ambiental, perteneciente a la DGRVA\\

\textbullet INEGI:Instituto Nacional de Estadística, Geografía e Informática\\

\textbullet LAUDF:Licencia Ambiental Única del Distrito Federal.\\

\textbullet NRA:Número de Registro Ambiental de un establecimiento\\

\textbullet RFC:Registro Federal de Causantes\\

\textbullet SIRS:Sistema de Información de Residuos Sólidos\\

\textbullet SMA:Secretaría del Medio Ambiente del Distrito Federal\\

\textbullet SOS:Secretaría de Obras y Servicios\\

\textbullet DGSU:Dirección General de Servicios Urbanos, perteneciente a la SOS\\


%--------------------------------------------------------------------
\section{referencias}
http://www.sedema.df.gob.mx/sedema/index.php/temas-ambientales/programas-generales/residuos-solidos

