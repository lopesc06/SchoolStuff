
%--------------------------------------------------------------------
\section{Objetivo general}
Desarrollar un sistema de información con el propósito de manejar y analizar de manera sistemática la información disponible en materia de residuos sólidos.

%--------------------------------------------------------------------
\section{Objetivos específicos}

\textbullet Facilitar el origen, destino, ruta y la cantidad de residuos sólidos de cada vehículo destinado al transporte de residuos.\\\\
\textbullet Permitir modificar, eliminar o crear nuevos reportes para el inventario de residuos.\\\\
\textbullet Posibilidad de inhabilitar de una manera temporal una ruta en específico de cierto transporte de residuos.\\\\
\textbullet Actualización de  ciertos catálogos que el sistema actualmente no soporta.\\\\

%--------------------------------------------------------------------
\section{Requerimientos del usuario}
{\large \textit{\textbf{\\1.-SuperUsuario}}}\\\\
\textbullet El usuario necesita Consultar \\

\textbullet Información de la captura de delegaciones

\textbullet Información de la captura de planes de manejo

\textbullet Información de la captura de infraestructura\\\\
\textbullet El usuario necesita dar de alta\\

\textbullet Información en los catálogos:

\textbullet Información de rutas nuevas

\textbullet Información de nuevos sitios de recolección

\textbullet Alta y baja de información de rutas existentes

\textbullet Alta y baja de información de sitios de recolección existentes

\textbullet Información de establecimientos nuevos

\textbullet Información de planes de manejo de establecimientos existentes

\textbullet Información de los apartados de un plan de manejo de un establecimiento existente.

\textbullet Alta y baja de información de infraestructura\\\\\\
\textbullet El usuario necesita modificar  la información de los catálogos:

\textbullet Rutas existentes

\textbullet Sitios de recolección existentes\\\\\\
\textbullet El usuario necesita consultar Informes

\textbullet Informes del módulo 1 (delegaciones)

\textbullet Informes del módulo 2 (planes de manejo)

\textbullet Informes del módulo 3 (infraestructura)\\\\\\
\textbullet El usuario necesita administrar
Usuarios

\textbullet Baja de rutas y sitios de recolección existentes

\textbullet Baja de un establecimiento existente

\textbullet Baja de un plan de manejo de un establecimiento existente\\\\\\
\textbullet El usuario necesita modificar

\textbullet La programación y de la base de datos.\\\\\\
{\large \textit{\textbf{2.-Administrador}}}\\\\\\
\textbullet El usuario necesita Consultar \\

\textbullet Información de la captura de delegaciones

\textbullet Información de la captura de planes de manejo

\textbullet Información de la captura de infraestructura\\\\
\textbullet El usuario necesita dar de alta\\

\textbullet Información en los catálogos:

\textbullet Información de rutas nuevas

\textbullet Información de nuevos sitios de recolección

\textbullet Alta y baja de información de rutas existentes

\textbullet Alta y baja de información de sitios de recolección existentes

\textbullet Información de establecimientos nuevos

\textbullet Información de planes de manejo de establecimientos existentes

\textbullet Información de los apartados de un plan de manejo de un establecimiento existente.

\textbullet Alta y baja de información de infraestructura\\\\\\
\textbullet El usuario necesita modificar  la información de los catálogos:

\textbullet Rutas existentes

\textbullet Sitios de recolección existentes\\\\\\
\textbullet El usuario necesita consultar Informes

\textbullet Informes del módulo 1 (delegaciones)

\textbullet Informes del módulo 2 (planes de manejo)

\textbullet Informes del módulo 3 (infraestructura)\\\\\\
\textbullet El usuario necesita administrar
Usuarios

\textbullet Baja de rutas y sitios de recolección existentes

\textbullet Baja de un establecimiento existente

\textbullet Baja de un plan de manejo de un establecimiento existente\\\\\\
{\large \textit{\textbf{3.-Supervisor del módulo 1}}}\\\\\\
\textbullet El usuario necesita Consultar \\

\textbullet Información de la captura de delegaciones\\\\\\
\textbullet El usuario necesita dar de alta\\

\textbullet Información en los catálogos:

\textbullet Información de rutas nuevas

\textbullet Información de nuevos sitios de recolección

\textbullet Alta y baja de información de rutas existentes

\textbullet Alta y baja de información de sitios de recolección existentes\\\\\\
\textbullet El usuario necesita modificar  la información de los catálogos:

\textbullet Rutas existentes

\textbullet Sitios de recolección existentes\\\\\\
\textbullet El usuario necesita consultar Informes\\
\textbullet Informes del módulo 1 (delegaciones)\\\\\\
{\large \textit{\textbf{4.-Capturista modulo 1}}}\\\\\\
\textbullet El usuario necesita Consultar \\

\textbullet Información de la captura de delegaciones\\\\\\
\textbullet El usuario necesita dar de alta\\

\textbullet Alta y baja de información de rutas existentes

\textbullet Alta y baja de información de sitios de recolección existentes\\\\\\
\textbullet El usuario necesita consultar Informes

\textbullet Informes del módulo 1 (delegaciones)\\\\\\
{\large \textit{\textbf{5.-Supervisor del modulo 2}}}\\\\\\
\textbullet El usuario necesita Consultar \\

\textbullet Información de la captura de planes de manejo\\\\
\textbullet El usuario necesita dar de alta\\

\textbullet Información de establecimientos nuevos

\textbullet Información de planes de manejo de establecimientos existentes

\textbullet Información de los apartados de un plan de manejo de un establecimiento existente.\\\\\\
\textbullet El usuario necesita consultar Informes

\textbullet Informes del módulo 2 (planes de manejo)\\\\\\
{\large \textit{\textbf{6.-Capturista del modulo 2}}}\\\\\\
\textbullet El usuario necesita Consultar 

\textbullet Información de la captura de planes de manejo\\\\
\textbullet El usuario necesita dar de alta

\textbullet Información de los apartados de un plan de manejo de un establecimiento existente.\\\\\\
\textbullet El usuario necesita consultar Informes

\textbullet Informes del módulo 2 (planes de manejo)\\\\\\
{\large \textit{\textbf{7.-Supervisor del modulo 3}}}\\\\\\
\textbullet El usuario necesita Consultar 

\textbullet Información de la captura de infraestructura\\\\
\textbullet El usuario necesita dar de alta

\textbullet Alta y baja de información de infraestructura\\\\\\
\textbullet El usuario necesita consultar Informes

\textbullet Informes del módulo 3 (infraestructura)\\\\\\
{\large \textit{\textbf{8.-Capturista del modulo 3}}}\\\\\\
\textbullet El usuario necesita Consultar 

\textbullet Información de la captura de infraestructura\\\\
\textbullet El usuario necesita dar de alta

\textbullet Alta y baja de información de infraestructura\\\\\\
\textbullet El usuario necesita consultar Informes

\textbullet Informes del módulo 3 (infraestructura)\\\\\\
{\large \textit{\textbf{9.-Usuario de consulta de perfil alto}}}\\\\\\
\textbullet El usuario necesita Consultar 

\textbullet Información de la captura de delegaciones

\textbullet Información de la captura de planes de manejo

\textbullet Información de la captura de infraestructura\\\\
\textbullet El usuario necesita consultar Informes

\textbullet Informes del módulo 1 (delegaciones)

\textbullet Informes del módulo 2 (planes de manejo)

\textbullet Informes del módulo 3 (infraestructura)\\\\\\
{\large \textit{\textbf{10.-Usuario de consulta de perfil bajo}}}\\\\\\
\textbullet El usuario necesita consultar Informes

\textbullet Informes del módulo 1 (delegaciones)

\textbullet Informes del módulo 2 (planes de manejo)

\textbullet Informes del módulo 3 (infraestructura)\\\\\\

%--------------------------------------------------------------------
\section{Requerimientos del negocio}

\textbullet El sistema debe ayudar a  tener un orden en la información manejada por la secretaria, evitando duplicidades e incoherencia en la misma.

\textbullet El sistema debe realizar informes de las acciones de la secretaría  según las necesidades.

\textbullet El sistema debe permitir una mayor agilidad y eficacia en los empleados ya que facilitará sus labores en la captura y manipulación de la información.

\textbullet El sistema debe permitir que la secretaría registre de manera diaria  la cantidad recolectada de residuos por cada viaje, especificando toneladas, tipo de residuo, vehículo, ruta y chofer.

\section{Requerimientos funcionales}

\textbullet El sistema  debe hacer la incorporación de los sistemas existentes y futuros, siempre que sean útiles para el funcionamiento adecuado del SIDAM.

\textbullet El sistema debe mostrar el  origen, destinos, ruta y la cantidad de residuos sólidos de cada vehículo buscando en el conjunto inicial de la base de datos o seleccionar un subconjunto de ella.

\textbullet El sistema debe hacer que la comunicación sea disponible, eficaz, integral, consistente y con un grado de usabilidad de la información. Para que este pueda brindar el servicio adecuado.

\section{Requerimientos de plataforma}


\subsection{Requerimientos de Hardware}
Computadora de escritorio o laptop con las siguientes especificaciones mínimas :\\

\textbullet Memoria RAM de 1GB.

\textbullet 500 MB de espacio libre en disco duro

\textbullet Procesador Intel Pentium 4 o superior
\subsection{Requerimientos de Software}
Navegador Web con las últimas actualizaciones.

\subsection{Red y otros servicios}

Acceso a internet con velocidad mínima de 3 Mbps de banda ancha.
\subsection{Interacción con otros sistemas}
EL SIRSDF debe de poder Coordinar la actualización, ejecución y evaluación del Programa de Gestión Integral de los Residuos Sólidos y mediante la sincronización de los datos y acceso a los demás sistemas como el SIDAM entre otros que llevan a cabo el manejo de la información relevante para el sistema, se generen los reportes sin necesidad de insertar nuevamente toda la información de manera manual al sistema SIRSDF.

\section{Requerimientos de interacción con el usuario}
La interfaz debe ser amigable, de fácil manejo para usuarios de poca experiencia con el apoyo de botones,cajas de textos, formularios, selección múltiple,listas desplegables.


\section{Datos e información que debe manejar el sistema}
\textbullet Nombre de la Estación de transferencia, planta de separación y planta de 	composta.

\textbullet Tipo de residuos.

\textbullet Rutas destinada.

\textbullet Tipo de formato de plan de trabajo.

\textbullet Nombre de la delegación y Colonias . 

\textbullet Toneladas de residuos.


\textbullet Modelo de Vehículo   del transporte.

\textbullet Número de matrícula de vehículo.

\textbullet Datos del personal

\section{Propiedades no funcionales del sistema}

% Liste las propiedades del proyecto completo sin importar que sea posible terminar durante el curso

%- - - - - - - - - - - - - - - - - - - - - - - - - - - - - - - - - - 
\subsection{Desempeño}
El portal web debe ser capaz de soportar el acceso de múltiples usuarios de manera simultánea sin verse afectado el rendimiento de este,haciendo creer al usuario que es el único que está teniendo uso de  él.
 
\subsection{Fiabilidad}
El sistema tendrá una base de datos y una gestión de la información en la cual se evitará tener datos innecesarios.Esto permitirá tener reportes confiables ya que solo los usuarios capacitados  y con acceso designado podrán agregar y modificar la información. Viéndose reflejado cualquier tipo de cambio en el sistema así como en sus relaciones, permitiendo coherencia a los datos del sistema.
\subsection{Disponibilidad}
El sistema web podrá ser accedido desde cualquier ordenador que cumpla con los requerimientos de plataforma; las 24 hrs del día,los 365 días del año.

\subsection{Seguridad}
El sistema web tendrá una gestión de acceso por usuario y contraseña con la cual se identifica que tipo de usuario es el que accede y se identificaran los privilegios que tiene este ya sean solo de visualización o en su caso de modificación de la información.

\subsection{Mantenibilidad}
El sistema será capaz de adaptarse a los cambios que vayan surgiendo a lo largo del ciclo de vida sin alterar su funcionamiento anterior y sin producir nuevos errores, añadiendo así nuevas funcionalidades que cubren nuevas necesidades.

\subsection{Portabilidad}
Debido a que el sistema es web, éste no necesita ser instalado por lo tanto no requiere de mayor esfuerzo al cambiar la plataforma desde donde se accede.
\section{Alcance para el curso}

\textbullet Inicio. Consistirá en una pantalla de bienvenida donde el usuario tendrá la posibilidad de acceder  con su cuenta que se ingresarán a través de cajas de texto para recibir la información.\\\\

\textbullet Edición.ciertas pantallas tendrán  campos  para la edición de la información mediante el uso de cajas de texto,listas desplegables,uso de opciones múltiples, 
En pantalla se  mostrará la información con respecto a todos  los transportes recolectores de sólidos en forma de tabla donde se verá la información más relevante de esta como son el número de matrícula,el conductor del equipo de transporte,las toneladas recolectadas,su origen y su destino más recientes, en caso de más información detallada,esta se desplegará en una ventana aparte.\\\\

\textbullet El sistema no contendrá comunicación con otros sistemas, por lo cual la inserción de la información dependerá de que los usuarios lo lleven a cabo de manera manual.\\\\

El sistema podrá ser utilizado a través de un navegador actualizado para poder generar formularios personalizados con la información recolectada y administrada por el sistema.
También manejará niveles de acceso en los cuales cada usuario tendrá diferentes funciones para manejar,el administrador del sistema asignará los privilegios para cada usuario registrado. 
Se tendrá un sistema de seguridad de acceso para que los usuarios con su nombre y contraseña previamente registrados por el administrador pueda acceder.
Se usará como servidor “servNet”, el cuales un tipo de hospedaje en la nube, el cual no requiere configurar OS, servidores o Bases de Datos antes del despliegue de datos, este servidor escala la aplicación arriba o abajo dependiendo la cantidad de usuarios, aumentando o disminuyendo la cantidad de memoria o de procesadores en uso, por lo tanto es adaptable, esto por un precio de $1,7172 al mes y dando un precio de $205344 al año.


