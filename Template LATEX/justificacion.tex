

%--------------------------------------------------------------------
\section{Antecedentes}
La Secretaría del Medio Ambiente del Distrito Federal cuenta con un sistema informático que maneja la información referente a los residuos sólidos.\\\\
El SIRS se desarrolló con el propósito de manejar y analizar, de manera sistemática, la información disponible en materia de residuos sólidos para apoyar acciones en la planeación, desarrollo de infraestructura de tratamiento, disposición de residuos y de investigación en el área de residuos sólidos, así como de proporcionar información confiable y actualizada a la ciudadanía a través del Inventario de Residuos Sólidos del Distrito Federal.\\\\
Los tres principales generadores de información en materia de residuos son:
16 delegaciones, donde estas se encargan de la recolección, transporte y almacenamiento temporal de los residuos a nivel domiciliario.\\\\
La Secretaría de Obras y Servicios encargada de la operación de las plantas de separación, composta estaciones de transferencia y disposición final de residuos.\\\\
La Secretaria del Medio Ambiente, que mediante otras instituciones, emite políticas, lineamientos, compila y publica inventarios de residuos, además de otorgar autorizaciones(licencia Ambiental Única y el Manifiesto de Impacto Ambiental) para planes de manejo de residuos.




%--------------------------------------------------------------------
\section{Descripción de la problemática}
El Software SIRS no lleva a cabo diferentes procesos que son indispensables para  la Secretaría del Medio Ambiente del Distrito Federal, entre estos procesos están:\\\\

\textbullet Generar un reporte que contenga información como: Total recolectado, por colonia, delegación, vehículo, etc. de forma diaria, mensual, semanal o anual.\\\\

\textbullet La ruta no está plenamente identificada con las colonias, unidades habitacionales, negocios, mercados, etc.\\\\

\textbullet El sistema no contempla que los vehículos de recolección transportan toda la basura a las estaciones de transferencia y, posteriormente, los residuos se trasladan en vehículos Transfer a sitios de disposición final.\\\\\\\\
También tiene los siguientes problemas \\\\

\textbullet Los reportes que genera el sistema están limitados y no permite su personalización.\\\\

\textbullet Existen problemas de consistencia en la información.\\\\

\textbullet El sistema no ofrece interacción con otros sistemas.\\\\

\textbullet Lo anterior hace que la Secretaría considere que el sistema deba rehacerse o mejorarse sustancialmente.\\

%--------------------------------------------------------------------
\section{Soluciones en el mercado}
Para la problemática que se nos presenta no existen soluciones en el mercado (aparte del SIRS que se mencionó anteriormente) ya que lo que se necesita es un sistema hecho a la medida para cubrir las necesidades de un grupo de usuarios en específico.


%--------------------------------------------------------------------
\section{Propuesta de solución}
Tras analizar la problemática se propone\\\\
\textbullet Mejorar el sistema actual partiendo de la estructura que actualmente ya maneja añadiendo los procesos con los que actualmente no cuenta descritos en la problemática.\\\\
\textbullet Unificar todas las tecnologías concorde a los otros sistemas para su posterior integración.


%--------------------------------------------------------------------
\section{Beneficios esperados}
\textbullet Ayudar a la secretaría del Medio Ambiente del Distrito Federal a tener una mejor gestión de la información sobre las recolecciones que se realizan dependiendo las delegaciones en las cuales se realizan generando informes personalizados.\\\\
\textbullet Reducir el tiempo y costo para la recopilación, transferencia y comunicación de información.\\\\
\textbullet Facilitar el proceso de toma de decisiones y elaboración de informes.\\\\
\textbullet Organizar la información para mejorar el proceso de acceso a la información pública, para la ciudadanía.\\\\
\textbullet Generar información de tal manera que permita cobrar impuestos a empresas y demás entidades identificadas como “grandes generadores de recursos”.\\\\

