% En esta sección considere solo lo que se va a comprometer durante el curso. Considere el calendario escolar como límite de tiempo.

%--------------------------------------------------------------------
\section{Descripción metodológica}
Para el desarrollo del sistema emplearemos una metodología de desarrollo ágil llamada Programación extrema, la cual consiste en realizar entregas cortas al cliente por pequeños lapsos de tiempo, se adapta a pequeños equipos y puede lograr un sistema de calidad en poco tiempo.

\section{Desglose de actividades}

La metodología de desarrollo estará compuesta de 4 principales actividades:\\\\
\textbullet \textbf{Planeación.}
Una vez identificados los requerimientos descritos por el cliente(historias de usuario) se le asignará una prioridad a cada requerimiento ya sea Alta o Baja, posteriormente se estimará el tiempo para cumplir dicho requerimiento.\\\\
\textbullet \textbf{Diseño.}
Las historias de usuario serán trasladados a tarjetas CRC(Clase-Responsabilidad-Colaborador) esto con la finalidad de identificar y organizar las clases orientadas a objetos con más facilidad.\\\\
\textbullet \textbf{Codificación.}
Primero las historias de usuario se programaran de manera individual haciendo pruebas unitarias antes de integrarse al sistema principal esto con la razón de que el desarrollador esté más seguro y conozca con exactitud lo que se tiene que hacer. Posterior a esto se llevará a cabo la programación en pares para que así haya mayor retroalimentación y uno compruebe el trabajo de otro. Al finalizar se hará el proceso de refactorización, esto con el fin de hacer más legible y mejorar el código fuente.\\\\
\textbullet \textbf{Pruebas.}
Como ya se había mencionado anteriormente se deben llevar a cabo pruebas unitarias antes de la codificación. También se deben realizar pruebas de aceptación, esta consiste en entregar un avance del sistema al cliente con las nuevas historias de usuario  ya integradas.

Estas 4 actividades de irán repitiendo constantemente hasta la entrega final del proyecto.

\section{Cronograma de actividades}


\section{Equipo de trabajo}
El equipo de trabajo está conformado por 4 integrantes que, trabajan de manera equitativa y democrática, participando todos de tal manera que cada uno expone su punto de vista,  colaborando en aspectos como el análisis(y todos los aspectos que tienen que ver con ello), diseño, implementación y desarrollo.\\
El equipo se comunica y desenvuelve plasmando las ideas y opiniones en una herramienta tecnológica llamada “Google Doc”, para posteriormente redactarlo en un documento Latex.Otra forma en la que el equipo se organizó fue durante sesiones en horas específicas dentro de la escuela, en la que se comentaban las ideas del proyecto, se lanzaban ideas al aire de tal manera que hubiese una retroalimentación, argumentando el porqué una idea pudiese ser mejor que otra.\\
También se cuenta con un líder de proyecto, que en general ayuda a organizarnos, proponiendo tiempos de trabajo, tecnologías en las cuales trabajar para un mejor manejo de las ideas.	

\section{Entregables}
\textbullet Manual Técnico.

\textbullet Manual de usuario del sistema considerando los diferentes tipos de usuario.

\textbullet Código fuente de todo el sistema.

\textbullet Base de datos (Modelo y Scripts).
\section{Resumen de tiempo y costo}
